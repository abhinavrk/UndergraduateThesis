%% This is an example first chapter.  You should put chapter/appendix that you
%% write into a separate file, and add a line \include{yourfilename} to
%% main.tex, where `yourfilename.tex' is the name of the chapter/appendix file.
%% You can process specific files by typing their names in at the 
%% \files=
%% prompt when you run the file main.tex through LaTeX.


\chapter{Conclusion and Recommendations}

In this thesis an existing design for a multi-mode Dielectric Grating based Surface Plasmon Resonance (DGSPR) sensor was optimized, with respect to production, operating considerations, and cost. To do the optimization, a simulation of the DGSPR structure was used, and the design's parameters were varied manually (within a constrained design space), keeping previously identified key manufacturing considerations in mind. Proceeding in such a manner, the 6 largest local maxima were found (within the constrained design space) and in doing so the global maxima (the largest of the local maxima) was also identified. 

Based on the optimized design, a prototype master mold was fabricated using a mixture of Electron-beam lithography and Deep Reactive Ion Etching (a Pseudo Bosch process). The e-beam process was optimized to ensure that the resist was exposed at the correct beam dosage, that the proximity error correction bias term could be approximated accurately, and to ensure that the patterned Silicon/ma-N structure spent enough time in the ma-D developer. The DRIE process was also affected by issues regarding the improper development of the exposed ma-N pattern and was also optimized to ensure that the stripping process was completed correctly so as to minimize artifacts in the final structure. The final DGSPR master mold was examined under a electron microscope to determine the efficacy of the E-beam/DRIE process. The mold was found to have sufficiently smooth side-walls, and to have dimensions within 2\% of the required values.

It is theoretically possible to fabricate a larger version of the prototype mold to be used to make DGSPR sensors. An alternative is to use a smaller mold to create a metal shim which can be affixed to a rotor and used to create larger DGSPR sensors. Another alternative is to start with a small DGSPR sensor and to use micro/nano-positioners to imprint the pattern carefully onto a larger silicon wafer coated with ma-N (or some other resist) which would then be etched using the wall-e process. Suppose that in each iteration you quadrupled the size of the pattern i.e. suppose in the first iteration the pattern is on a $100 nm \times 100 nm$ area, then in the second iteration it would expand to over a $200 nm \times 200 nm$ square. Such a system would give you exponential growth and would be easiest way to maximize the use of the small resist. Over the course of sixteen iterations a $100 nm \times 100 nm$ area can grow to an $1 inch \times 1 inch$ area. The issue here is that the micro/nano-positioners have to be accurate enough to ensure that neighboring patterns do not interfere with each other. Asymptotically, if the system is sufficiently large and care is taken to ensure that the initial mold is free of any flaws, the significance of said errors should drop. Alternatively, a mix of the shim and the exponential growth idea could be used. This would definitely minimize the density of flaws in the large system. Such a system would be extremely useful for investigating statistical effects of repeated flaws (given the relatively constant density of flaws) and could be useful in the field of statistical physics/plasmonics. It would also simulate a `doped' grating structure. The specific uses of this are not currently known especially since the field of quantum statistics in surface plasmon polaritons \cite{quantum} is somewhat embryonic. 